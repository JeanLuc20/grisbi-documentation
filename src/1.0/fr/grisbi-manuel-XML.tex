%%%%%%%%%%%%%%%%%%%%%%%%%%%%%%%%%%%%%%%%%%%%%%%%%%%%%%%%%%%%%%%%%
% Contents: The XML chapter
% $Id: grisbi-manuel-XML.tex, v 0.4 2002/10/27 Daniel Cartron
%%%%%%%%%%%%%%%%%%%%%%%%%%%%%%%%%%%%%%%%%%%%%%%%%%%%%%%%%%%%%%%%%

\chapter{Format d'enregistrement du fichier de comptes\label{xml}}

\section{Généralités\label{xml-general}}

Le fichier de comptes de Grisbi est enregistré au format XML. Il est 
visualisable avec n'importe quel éditeur, mais il est très fortement recommandé
de ne pas le modifier.

Le format XML est un format universel, ce qui permet à n'importe quel autre 
programme de pouvoir utiliser les données provenant de Grisbi, à condition qu'il 
reprenne le formatage de Grisbi. C'est pourquoi nous allons détailler ce 
formatage ci-après.

Un fichier XML stocke des informations de manière hiérarchique. On reconnait
un fichier XML à son en-tête: \xml{<?xml version="1.0"?>}.

Les données étant organisées de manière arborescente, on distingue des n{\oe}uds de 
l'arbre, se divisant en plusieurs branches, et des feuilles, qui sont des n{\oe}uds 
finals contenant l'information recherchée.

La manière dont ces n{\oe}uds se suivent et sont imbriqués est libre, mais on
peut contraindre la grammaire d'un fichier XML en fonction d'une DTD
(Document Type Definition, ou Définition de Type de Document), qui dictera
les combinaisons autorisées.

\section{Vue d'ensemble}

Le principe du XML est l'utilisation de balises, un peu comme le HTML, sauf
qu'en XML c'est le programmeur qui définit ses propres balises.

Une information est entourée par deux balises, la balise ouvrante qui ressemble
à \cmd{<balise>} et la balise fermante qui ressemble à \cmd{/<balise>}.

Par exemple le numéro de version utilisé pour enregistrer votre fichier s'écrit~:

 \cmd{<Version>0.3.2/<Version>}.

Les informations sont classées en sections. Les sections débutent et finissent 
par des balises portant leur nom. Exemple~: les informations de configuration 
générale sont comprises entre les balises \cmd{<Generalites>} et 
\cmd{/<Generalites>}.

Si une balise ne contient aucune information le XML écrit une balise
vide sous la forme \cmd{<balise/>}.

Dans la présentation de la structure du fichier de compte de Grisbi nous 
noterons "/" le fait qu'un élément est compris dans un autre.

Certaines balises ont un nom tellement explicite que nous nous contenterons de 
les citer sans autres explications.

\section{Organisation du fichier}

\subsection{Section de la configuration générale\label{xml-setup}}

Contenu de cette section~:

\begin{itemize}

\item \xml{<Grisbi>}~: c'est l'élément racine, ancêtre de tous les autres~;

\item \xml{<Grisbi>/<Generalites>}~: le bloc d'informations générales
relatives à la configuration de Grisbi~;

\item \xml{<Grisbi>/<Generalites>/<Version\_fichier>}~: le numéro de version 
du format de fichier. Ce numéro peut être différent du numéro de version de Grisbi~;

\item \xml{<Grisbi>/<Generalites>/<Version\_Grisbi>}~: le numéro de version de Grisbi~;

\item \xml{<Grisbi>/<Generalites>/<Fichier\_ouvert>}~: est utilisée pour le
verrouillage dans le mode multi-utilisateurs. 1~=~le fichier est ouvert, 
0~=~le fichier n'est pas ouvert ~;

\item \xml{<Grisbi>/<Generalites>/<Backup>}~: chemin et nom du fichier de 
sauvegarde~;

\item \xml{<Grisbi>/<Generalites>/<Titre>}~: le titre de votre fichier~;

\item \xml{<Grisbi>/<Generalites>/<Adresse\_commune>}~: l'adresse du titulaire du
compte~;

\item \xml{<Grisbi>/<Generalites>/<Adresse\_secondaire>}~: une deuxième zone 
adresse pour enregistrer par exemple l'adresse du trésorier de l'association~;

\item \xml{<Grisbi>/<Generalites>/<Utilise\_exercices>}~: 1~=~le fichier de compte
utilise les exercices, 0~=~il ne les utilise pas~;

\item \xml{<Grisbi>/<Generalites>/<Utilise\_IB>}~: 1~=~le fichier de compte
utilise les imputations budgétaires, 0~=~il ne les utilise pas~;

\item \xml{<Grisbi>/<Generalites>/<Utilise\_PC>}~: 1~=~le fichier de compte
utilise le champ \cmd{Pièces comptables}, 0~=~il ne l'utilise pas~;

\item \xml{<Grisbi>/<Generalites>/<Utilise\_info\_BG>}~: 1~=~le fichier de compte
utilise le champ \menu{Coordonnées bancaires}, 0~=~il ne l'utilise pas~;

\item \xml{<Grisbi>/<Generalites>/<Numero\_devise\_totaux\_tiers>}~: le numéro de
la devise utilisée pour les totalisations dans les onglets \menu{Tiers}
\menu{Catégories} et \menu{Imputations budgétaires}~;

\item \xml{<Grisbi>/<Generalites>/<Type\_affichage\_des\_echeances>}~: 0~=~mois,
1~=~2 mois, 2~=~année, 3~=~toutes, 4~=~personnalisée~;

\item \xml{<Grisbi>/<Generalites>/<Affichage\_echeances\_perso\_nb\_libre>}~: 
contenu de l'entrée pour l'affichage personnalisé des échéances~;

\item \xml{<Grisbi>/<Generalites>/<Type\_affichage\_perso\_echeances>}~: 0~=~
jours, 1~=~mois, 2~=~années~;

\item \xml{<Grisbi>/<Generalites>/<Numero\_derniere\_operation>}~: numéro de la 
dernière opération du fichier~;

\item \xml{<Grisbi>/<Generalites>/<Chemin\_logo>}~: le chemin d'accès au logo utilisé dans le fichier~; 

\item \xml{<Grisbi>/<Generalites>/<Affichage\_opes>}~: suite de 28 numéros séparés par des tirets déterminant l'affichage des informations dans l'onglet \menu{Opérations}.  Les 28 numéros représentent les quatre lignes, chacune de 7 colonnes. Les numéros des informations sont~:

	\begin{itemize}

	\item 1~=~Date de l'opération~;

	\item 2~=~Date de valeur~;

	\item 3~=~Tiers~;

	\item 4~=~Imputation budgétaire~;

	\item 5~=~Débit~;

	\item 6~=~Crédit~;

	\item 7~=~Solde~;

	\item 8~=~Montant en devise~;

	\item 9~=~Moyen de paiement~;

	\item 10~=~Numéro de rapprochement~;

	\item 11~=~Exercice~;

	\item 12~=~ Catégorie~;

	\item 13~=~P/R (opération pointée ou rapprochée)~;

	\item 14~=~Pièce comptable~;

	\item 15~=~Notes~;

	\item 16~=~Informations bancaires~;

	\item 17~=~Numéro de l'opération~;

	\item 18~=~Numéro de chèque.

	\end{itemize} 

Un zéro signifie qu'aucune information n'est affichée dans la cellule. Le paramétrage par défaut est~:
18-1-3-13-5-6-7-0-0-12-0-9-8-0-0-11-15-0-0-0-0-0-0-0-0-0-0-0

\item \xml{<Grisbi>/<Generalites>/<Rapport\_largeur\_col>}~: suite de 7 nombres séparés par des tirets, représentant la variation de la taille des 7 colonnes de l'onglet \menu{Opérations} en pourcentage (par défaut 11-13-30-3-11-11-11)~;

\item \xml{<Grisbi>/<Generalites>/<Ligne\_aff\_une\_ligne>}~: le numéro (de 0 à 3) de la ligne affichée lors de l'affichage des opérations sur une seule ligne dans l'onglet \menu{Opérations} (par défaut 0)~;

\item \xml{<Grisbi>/<Generalites>/<Lignes\_aff\_deux\_lignes>}~: les deux numéros (de 0 à 3), séparés par un tiret, des lignes affichées lors de l'affichage des opérations sur deux lignes dans l'onglet \menu{Opérations} (par défaut 0-1)~;

\item \xml{<Grisbi>/<Generalites>/<Lignes\_aff\_trois\_lignes>}~: les trois numéros (de 0 à 3), séparés par des tirets, des lignes affichées lors de l'affichage des opérations sur trois lignes dans l'onglet \menu{Opérations} (par défaut 0-1-3).

\end{itemize}

\subsection{Section relative aux comptes}

Cette section est une branche de l'arbre Grisbi et devrait donc
être notée

 \xml{<Grisbi>/<Comptes>}.

Afin de simplifier nous n'indiquerons pas l'élément racine \xml{<Grisbi>}
et nous commencerons à \xml{<Comptes>}. Cette remarque est valable pour toutes 
les sections qui suivront.

Voici le contenu de cette section~:

\begin{itemize}

\item \xml{<Comptes>/<Generalites>} est une sous-section qui décrit les points 
communs à tous les comptes du fichier~;

\item \xml{<Comptes>/<Generalites>/<Ordre\_des\_comptes>} est l'ordre dans
lequel les comptes sont affichés dans l'onglet \menu{Opérations} (configuré dans 
\menu{Configuration/Affichage}, voir le paragraphe \vref{setup-display}).
Les comptes sont désignés par leur numéro (voir ci-après) et séparés par des
tirets sans espace (exemple~: 3-2-0-1)~;

\item \xml{<Comptes>/<Generalites>/<Compte\_courant>}~: c'est le compte qui 
était affiché dans l'onglet \menu{Opérations} lors de la fermeture du fichier et 
qui le sera lors de sa ré-ouverture~;

\item \xml{<Comptes>/<Compte>} est un bloc d'informations renfermant tous les 
détails d'un compte donné. L'élément père \xml{<Comptes>} peut contenir 
plusieurs fils \xml{<Compte>} qui se suivent les uns les autres. Ils seront 
identifiés par leur numéro interne. Nous allons décrire ici la syntaxe attendue 
d'un tel compte.

\item \xml{<Comptes>/<Compte>/<Details>} ~: il s'agit de toutes les informations 
que vous avez saisies lors de la création du compte (ou peut-être après)~:

\item \xml{<Comptes>/<Compte>/<Details>/<Nom>}~: le nom que vous avez donné à 
votre compte~;

\item \xml{<Comptes>/<Compte>/<Details>/<No\_de\_compte >}~: le numéro de votre 
compte dans Grisbi~;

\item \xml{<Comptes>/<Compte>/<Details>/<Titulaire>}~: si  le titulaire du 
compte est différent du titulaire général~;

\item \xml{<Comptes>/<Compte>/<Details>/<Type\_de\_compte>}~: le type de compte 
( 0~=~bancaire,  1~=~espèces,  2~=~passif, 3~=~actif )~;

\item \xml{<Comptes>/<Compte>/<Details>/<Nb\_operations>}~: le nombre 
d'opérations enregistrées dans ce compte~;

\item \xml{<Comptes>/<Compte>/<Details>/<Devise>} ~: le numéro de la devise dans 
laquelle est tenu le compte~;

\item \xml{<Comptes>/<Compte>/<Details>/<Banque>} ~: le numéro de l'établissement 
bancaire tenant le compte~;

\item \xml{<Comptes>/<Compte>/<Details>/<Guichet>} ~: le code Guichet de 
l'établissement bancaire~;

\item \xml{<Comptes>/<Compte>/<Details>/<No\_compte\_banque>}~: le numéro de 
votre compte dans votre établissement bancaire~;

\item \xml{<Comptes>/<Compte>/<Details>/<Cle\_du\_compte>}~: la clé RIB~;

\item \xml{<Comptes>/<Compte>/<Details>/<Solde\_initial>}~;

\item \xml{<Comptes>/<Compte>/<Details>/<Solde\_mini\_voulu>}~;

\item \xml{<Comptes>/<Compte>/<Details>/<Solde\_mini\_autorise>}~;

\item \xml{<Comptes>/<Compte>/<Details>/<Solde\_courant>}~: pour accélérer les 
traitements Grisbi enregistre le solde de votre compte dans cette balise~;

\item \xml{<Comptes>/<Compte>/<Details>/<Date\_dernier\_releve>} ~: concerne les 
rapprochements~;

\item \xml{<Comptes>/<Compte>/<Details>/<Solde\_dernier\_releve>}~: concerne les 
rapprochements~;

\item \xml{<Comptes>/<Compte>/<Details>/<Dernier\_no\_de\_rapprochement>}~: 
concerne les rapprochements~;

\item \xml{<Comptes>/<Compte>/<Details>/<Compte\_cloture>}~: 0~=~Non, 1~=~Oui~;

\item \xml{<Comptes>/<Compte>/<Details>/<Affichage\_r>}~: si vous avez choisi d'avoir des caractéristiques d'affichage propres à chaque compte, cette balise détermine l'affichage des opérations rapprochées. 0~=~les opérations ne sont pas affichées, 1~=~elles le sont~;

\item \xml{<Comptes>/<Compte>/<Details>/<Nb\_lignes\_ope>}~: si vous avez choisi d'avoir des caractéristiques d'affichage propres à chaque compte, cette balise détermine le nombre de lignes affichées pour les opérations~;

\item \xml{<Comptes>/<Compte>/<Details>/<Commentaires>}~;

\item \xml{<Comptes>/<Compte>/<Details>/<Adresse\_du\_titulaire>}~;

\item \xml{<Comptes>/<Compte>/<Details>/<Nombre\_de\_types>}~: nombre de modes de paiement définis pour ce compte~;

\item \xml{<Comptes>/<Compte>/<Details>/<Type\_defaut\_debit>}~: le numéro du 
mode de paiement par défaut en débit~;

\item \xml{<Comptes>/<Compte>/<Details>/<Type\_defaut\_credit>}~: le numéro du 
mode de paiement par défaut en crédit~;

\item \xml{<Comptes>/<Compte>/<Details>/<Tri\_par\_type>}~: 0~=~tri par date 
lors du rapprochement, 1~=~tri par type d'opération~;

\item \xml{<Comptes>/<Compte>/<Details>/<Neutres\_inclus>}~: 1~=~les modes de paiement
neutres sont séparés en fonction du montant de l'opération (débit/crédit)~;

\item \xml{<Comptes>/<Compte>/<Details>/<Ordre\_du\_tri>}~: classement des modes
de paiement lors du rapprochement .

\end{itemize}

\subsubsection{Les types d'opérations du compte}

La balise suivante est une sous-sous-section qui se nomme 

\xml{<Comptes>/<Compte>/<Detail\_de\_Types>}. 

Elle décrit tous les types 
d'opérations paramétrés pour le compte. Le format change, à savoir que chaque 
type est contenu dans sa propre balise, incluant tous les paramètres le 
distinguant. La valeur des paramètres leur est affectée de la façon suivante~: 
\xml{Nom="Virement"} c'est-à-dire le nom du paramètre, le signe =, puis la 
valeur du paramètre entre guillemets. Un type d'opération est décrit par la
balise \xml{<Type>} contenant les paramètres suivants, séparés par des espaces~:

\begin{itemize}

\item \xml{<No>}~: numéro du type d'opération pour ce compte~;

\item \xml{<Nom>}~: le nom que vous donnez à ce type d'opération~;

\item \xml{<Signe>}~: 0=neutre, 1=débit, 2=crédit~;

\item \xml{<Affiche\_entree>}~: affiche l'entrée complémentaire (numéro de 
chèque ou de virement)~;

\item \xml{<Numerotation\_auto>}~: 1~=~incrémente le numéro en cours
lors de l'affichage de l'entrée complémentaire, 0~=~ne l'incrémente pas~;

\item \xml{<No\_en\_cours>}~: numéro en cours pour l'entrée complémentaire.

\end{itemize}

\subsubsection{Détail des opérations du compte}

La balise suivante est une sous-sous-section qui se nomme
\xml{<Comptes>/<Compte>/<Detail\_des\_operations>} et qui liste toutes les 
opérations du compte, ainsi que leurs lignes de ventilation. De même que les 
types, les opérations sont contenues dans une balise \xml{<Operation>} avec les 
paramètres suivants~:

\begin{itemize}

\item \xml{<No>}~: c'est le numéro de l'opération dans le fichier, tous comptes
confondus~;

\item \xml{<D>}~: la date de l'opération~;

\item \xml{<Db>}~: la date de valeur de l'opération~;

\item \xml{<M>}~: le montant (signé) de l'opération. A noter que bien que le 
formulaire de saisie présente deux zones, une pour les débits et une pour les 
crédits, le montant est enregistré dans un seul paramètre~;

\item \xml{<De>}~: le numéro de la devise utilisée~;

\item \xml{<Rdc>}~: c'est le Rez-de-chaussée. Heu, non, c'est l'indication de la 
devise à laquelle on applique le taux de change défini plus loin~:

\begin{itemize}

\item 1~: devise du compte~=~devise étrangère x taux de change~;

\item 0~: devise étrangère~=~devise du compte  x taux de change.

\end{itemize}

\item \xml{<Tc>}~: le taux de change permettant de passer de la devise du compte 
à la devise étrangère ou inversement~;

\item \xml{<Fc>}~: les frais de change, s'il y en a~;

\item \xml{<T>}~: le numéro du tiers~;

\item \xml{<C>}~: le numéro de la catégorie~;

\item \xml{<Sc>}~: le numéro de la sous-catégorie~;

\item \xml{<Ov>}~: 0~=~l'opération n'est pas ventilée, 1~=~l'opération est
ventilée~;

\item \xml{<N>}~: les notes~;

\item \xml{<Ty>}~: le numéro du type de l'opération tel que définit plus haut~;

\item \xml{<Ct>}~: l'information liée au type d'opération (n° de chèque, de 
virement,
etc.)~;

\item \xml{<P>}~: le statut de l'opération~: 0~=~non pointée, 1~=~pointée mais 
non 
rapprochée, 2~=~rapprochée~;

\item \xml{<A>}~: si l'opération est planifiée, définit son statut~: 1~=~
automatique, 0
= manuelle~;

\item \xml{<R>}~: le numéro du rapprochement~;

\item \xml{<E>}~: le numéro de l'exercice~;

\item \xml{<I>}~: le numéro de l'imputation budgétaire~;

\item \xml{<Si>}~: le numéro de la sous-imputation budgétaire~;

\item \xml{<Pc>}~: les références de la pièce comptable~;

\item \xml{<Ibg>}~: les informations Banque - Guichet~;

\item \xml{<Ro>}~: pour les virements, le numéro de l'opération de contrepartie 
;

\item \xml{<Rc>}~: pour les virements, le numéro du compte de contrepartie~;

\item \xml{<Va>}~: si l'opération est une ligne de ventilation, ce paramètre 
contient le 
numéro de l'opération-mère.

\end{itemize}

\subsection{Échéances}

Cette section traite des opérations planifiées et comprend les informations
suivantes~:

\begin{itemize}

\item \xml{<Echeances>/<Generalites>}~: la sous-section des généralités 
comprenant~:

\item \xml{<Echeances>/<Generalites>/<Nb\_echeances>}~: le nombre d'opérations 
planifiées~;

\item \xml{<Echeances>/<Generalites>/<No\_derniere\_echeance>}~: le numéro de la 
dernière opération planifiée~;

\item \xml{<Echeances>/<Detail\_des\_echeances>}~: Cette balise ouvre la liste
des opérations planifiées décrites dans une balise nommée 
\xml{<Echeances>/<Detail\_des\_echeances>/<echeance>}. Les paramètres
de cette balise correspondent aux champs à renseigner lors de la création de 
l'opération planifiée~:

\begin{itemize}

\item \xml{<No>}~: numéro d'ordre de l'opération planifiée~;

\item \xml{<Date>}~: date de la première échéance de l'opération 
planifiée~;

\item \xml{<Compte>}~: compte concerné par l'opération planifiée~;

\item \xml{<Montant>}~;

\item \xml{<Devise>}~;

\item \xml{<Tiers>}~;

\item \xml{<Categorie>}~;

\item \xml{<Sous-categorie>}~;

\item \xml{<Virement\_compte>}~: numéro de compte vers lequel ou duquel le virement
s'effectue~;

\item \xml{<Type>}~;

\item \xml{<Contenu\_du\_type>}~: lorsqu'un type d'opération affiche une entrée
complémentaire, mais non incrémentée automatiquement, le contenu de cette entrée 
est enregistré ici~;

\item \xml{<Exercice>}~;

\item \xml{<Imputation>}~: il s'agit de l'imputation budgétaire~;

\item \xml{<Sous-imputation>}~: voir ligne précédente~;

\item \xml{<Piece\_comptable>}~;

\item \xml{<Info\_banque\_guichet>}~;

\item \xml{<Notes>}~;

\item \xml{<Automatique>}~: 0~=~manuelle, 1~=~automatique~;

\item \xml{<Periodicite>}~: 0=une fois, 1=hebdo, 2=mensuel, 3=annuel, 
4=personnalisé~;

\item \xml{<Intervalle\_periodicite>}~: 0=jours, 1=mois, 2=années~;

\item \xml{<Periodicite\_personnalisee>}~: le nombre de jours/mois/années quand 
la périodicité est personnalisée~;

\item \xml{<Date\_limite>}.

\end{itemize}

\end{itemize}

\subsection{Tiers}

Cette section liste tous les tiers enregistrés dans le fichier de comptes~:

\begin{itemize}

\item \xml{<Tiers>/<Generalites>}~: cette sous-section contient comme d'habitude
le nombre de tiers et le numéro du dernier tiers enregistré~;

\item \xml{<Tiers>/<Generalites>/<Nb\_tiers>}~;

\item \xml{<Tiers>/<Generalites>/<No\_dernier\_tiers>}~;

\item \xml{<Tiers>/<Detail\_des\_tiers>}~: cette sous-section comprend autant de 
balises \xml{<Tiers>} que de tiers enregistrés. Chaque balise décrit un tiers 
avec les paramètres suivants~:

\begin{itemize}

\item \xml{<No>}~;

\item \xml{<Nom>}~;

\item \xml{<Informations>}~;

\item \xml{<Liaison>}~: 1~=~le tiers est lié à un autre logiciel, 0~=~le tiers 
n'est pas lié (sans intérêt pour l'instant).

\end{itemize}

\end{itemize}

\subsection{Catégories\label{xml-categories} }

Cette section est construite exactement sur le même principe que celle des tiers 
et comprend~:

\begin{itemize}

\item \xml{<Categories>/<Generalites>}~;

\item \xml{<Categories>/<Generalites>/<Nb\_Categories>}~;

\item \xml{<Categories>/<Generalites>/<No\_derniere\_categorie>}~;

\item \xml{<Categories>/<Detail\_des\_categories>}~: sous-section comportant les 
balises \xml{<Categorie>} dont chacune décrit une  catégorie~:

\begin{itemize}

\item \xml{<No>}~;

\item \xml{<Nom>}~;

\item \xml{<Type>}~: 1 indique que cette catégorie est normalement au débit, 0 
au crédit et 2 signifie spécial (concrètement, c'est une catégorie de virement 
ou d'opération ventilée)~;

\item \xml{<No\_derniere\_sous\_categorie>}.

\end{itemize}

\item \xml{<Categories>/<Sous\_categorie>}~: si la catégorie contient une ou 
plusieurs sous-catégories la balise \xml{<Categorie>} est suivie par une ou
plusieurs balises \xml{<Sous\_categorie>} contenant les paramètres suivants~:

\begin{itemize}

\item \xml{<No>}~: le numéro d'ordre de la sous-catégorie dans la catégorie (de
1 à n, n étant normalement égal au contenu de la balise 
\xml{<No\_derniere\_sous\_categorie>}, sauf si la dernière sous-catégorie a été 
supprimée~;

\item \xml{<Nom>}.

\end{itemize}

\end{itemize}

\subsection{Imputations budgétaires\label{xml-budgetlines} }

Cette section est exactement identique à celle des catégories, à ceci près que le 
mot catégorie est remplacé par le mot imputation~:

\begin{itemize}

\item \xml{<Imputations>/<Generalites>}~: cette sous-section contient le nombre 
d'imputations et le numéro de la dernière imputation enregistrée~;

\item \xml{<Imputations>/<Generalites>/<Nb\_imputations>}~;

\item \xml{<Imputations>/<Generalites>/<No\_derniere\_imputation>}~;

\item \xml{<Categories>/<Detail\_des\_imputations>}~: sous-section comportant 
les balises \xml{<Imputation>} dont chacune décrit une imputation~:

\begin{itemize}

\item \xml{<No>}~: le numéro d'ordre de l'imputation~;

\item \xml{<Nom>}~;

\item \xml{<Type>}~;

\item \xml{<No\_derniere\_sous\_imputation>}.

\end{itemize}

\item \xml{<Categories>/<Sous\_imputation>}~: si l'imputation contient une ou 
plusieurs sous-imputations la balise \xml{<Imputation>} est suivie par une ou
plusieurs balises \xml{<Sous\_imputation>} contenant les paramètres suivants~:
\begin{itemize}

\item \xml{<No>}~: le numéro d'ordre de la sous-imputation dans l'imputation (de
1 à n, n étant normalement égal au contenu de la balise 
\xml{<No\_derniere\_sous\_imputation>}, sauf si la dernière sous-imputation a 
été supprimée)~;

\item \xml{<Nom>}.

\end{itemize}

\end{itemize}

\subsection{Devises\label{xml-currencies} }

La gestion des devises est antérieure au passage à la monnaie unique, ce qui
explique qu'il reste un certain nombre de paramètres liés au passage à l'euro. 
Ces paramètres seront conservés encore quelque temps pour permettre l'import 
d'anciens fichiers, puis disparaîtront.

La section \xml{<Devises>} utilise la structure désormais familière des autres
sections du fichier~:

\begin{itemize}

\item \xml{<Devises>/<Generalites>}~: la sous-section des généralités, à savoir
:

\item \xml{<Devises>/<Generalites>/<Nb\_devises>}~;

\item \xml{<Devises>/<Generalites>/<No\_derniere\_devise>}.

\item \xml{<Devises>/<Generalites>/<Detail\_des\_devises>}~: la sous-section 
des détails, comprenant des balises  \xml{<Devise>} et leurs paramètres~:

\begin{itemize}

\item \xml{<No>}~;

\item \xml{<Nom>}~;

\item \xml{<Code>}~;

\item \xml{<Passage\_euro>}~: 1~=~la devise fait partie de la zone euro, 0~=~elle ne fait
pas partie de la zone euro~;

\item \xml{<Date\_dernier\_change>}~: la date du dernier taux de change saisi~;

\item \xml{<Rapport\_entre\_devises>}~: 1~=~le taux de la devise est égal à la 
devise en 
rapport multipliée par le taux de change, 0~=~pas de conversion~;

\item \xml{<Devise\_en\_rapport>}~: la devise en rapport avec la devise courante 
(ex: liaison franc-euro)~;

\item \xml{<Change>}~: taux de change entre les deux devises.

\end{itemize}

\end{itemize}

\subsection{Banques\label{xml-bank} }

Nous retrouvons la structure habituelle~:

\begin{itemize}

\item \xml{<Banques>}~;

\item \xml{<Banques>/<Generalites>}~;

\item \xml{<Banques>/<Generalites>/<Nb\_banques>}~;

\item \xml{<Banques>/<Generalites>/<No\_derniere\_banque>}~;

\item \xml{<Banques>/<Detail\_des\_banques>}~;

\begin{itemize}

\item \xml{<No>}~;

\item \xml{<Nom>}~;

\item \xml{<Code>}~: c'est le code banque de votre RIB~;

\item \xml{<Adresse>}~;

\item \xml{<Tel>}~;

\item \xml{<Mail>}~;

\item \xml{<Web>}~;

\item \xml{<Nom\_correspondant>}~;

\item \xml{<Fax\_correspondant>}~;

\item \xml{<Tel\_correspondant>}~;

\item \xml{<Mail\_correspondant>}~;

\item \xml{<Remarques>}.

\end{itemize}

\end{itemize}

\subsection{Exercices\label{xml-financialyear} }

Cette section s'étoffera probablement lors de l'introduction des budgets. Pour 
l'instant elle comporte~:

\begin{itemize}

\item \xml{<Exercices>/<Generalites>}~;

\item \xml{<Exercices>/<Nb\_exercices>}~;

\item \xml{<Exercices>/<No\_dernier\_exercice>}~;

\item \xml{<Exercices>/<Detail\_des\_exercices>}~:

\begin{itemize}

\item \xml{<No>}~;

\item \xml{<Nom>}~;

\item \xml{<Date\_debut>}~;

\item \xml{<Date\_fin>}~;

\item \xml{<Affiche>}~: 1~=~l'exercice est affiché dans la liste déroulante du 
formulaire des opérations, 0~=~il n'est pas affiché.

\end{itemize}

\end{itemize}

\section{Rapprochements\label{xml-reconciliation} }

Grisbi permettant de personnaliser le numéro des rapprochements, cette section 
enregistre simplement la liste des numéros personnalisés~:

\xml{<Rapprochements>}~;

\xml{<Rapprochements>/<Detail\_des\_rapprochements>}~;

\begin{itemize}

\item \xml{<No>}~: ce numéro est repris dans la balise \xml{<Operation>/<R>}~;

\item \xml{<Nom>}~: le numéro (ou nom) personnalisé.

\end{itemize}

\section{États\label{xml-reports} }

Cette section enregistre le paramétrage de tous vos états.
Nous retrouvons ici aussi la structure familière des autres sections. 

\begin{itemize}

\item  \xml{<Etats>/<Generalites>} est une sous-section commune à tous les états, et qui ne comporte pour l'instant qu'une seule information~;

\item  \xml{<Etats>/<Generalites>/<No\_dernier\_etat>} indique le numéro du dernier état~;

\item  \xml{<Etats>/<Detail\_des\_etats>} est une sous-section enregistrant le détail de tous les états créés~;

      \item  \xml{<Etats>/<Detail\_des\_etats>/<Etat>} marque le début d'un état~;

        \item  \xml{<Etats>/<Detail\_des\_etats>/<Etat>/<No>}~: c'est le numéro d'ordre de l'état dans votre fichier~;

        \item  \xml{<Etats>/<Detail\_des\_etats>/<Etat>/<Nom>} le nom que vous avez donné à votre état~;

        \item  \xml{<Etats>/<Detail\_des\_etats>/<Etat>/<Type\_classement>}~: organisation des niveaux de regroupement (par défaut 1/2/3/4/5/6)~:

		\begin{itemize}

		\item  1~=~catégorie~;

		\item  2~=~sous-catégorie~;

		\item  3~=~imputation budgétaire~;

		\item  4~=~sous-imputation budgétaire~;

		\item  5~=~compte~;

		\item  6~=~tiers.

		\end{itemize} 

        \item  suivent 17 balises déterminant l'affichage des opérations. Une valeur 0 indique que l'information n'est pas affichée, une valeur 1 que l'information est affichée~:

		\begin{itemize}
		
		\item  \xml{<Etats>/<Detail\_des\_etats>/<Etat>/<Aff\_r>}

		\item  \xml{<Etats>/<Detail\_des\_etats>/<Etat>/<Aff\_ope>}

		\item  \xml{<Etats>/<Detail\_des\_etats>/<Etat>/<Aff\_nb\_ope>}

		\item  \xml{<Etats>/<Detail\_des\_etats>/<Etat>/<Aff\_no\_ope>}

		\item  \xml{<Etats>/<Detail\_des\_etats>/<Etat>/<Aff\_date\_ope>}

		\item  \xml{<Etats>/<Detail\_des\_etats>/<Etat>/<Aff\_tiers\_ope>}

		\item  \xml{<Etats>/<Detail\_des\_etats>/<Etat>/<Aff\_categ\_ope>}

		\item  \xml{<Etats>/<Detail\_des\_etats>/<Etat>/<Aff\_ss\_categ\_ope>}

		\item  \xml{<Etats>/<Detail\_des\_etats>/<Etat>/<Aff\_type\_ope>}

		\item  \xml{<Etats>/<Detail\_des\_etats>/<Etat>/<Aff\_ib\_ope>}

		\item  \xml{<Etats>/<Detail\_des\_etats>/<Etat>/<Aff\_ss\_ib\_ope>}

		\item  \xml{<Etats>/<Detail\_des\_etats>/<Etat>/<Aff\_cheque\_ope>}

		\item  \xml{<Etats>/<Detail\_des\_etats>/<Etat>/<Aff\_notes\_ope>}

		\item  \xml{<Etats>/<Detail\_des\_etats>/<Etat>/<Aff\_pc\_ope>}

		\item  \xml{<Etats>/<Detail\_des\_etats>/<Etat>/<Aff\_rappr\_ope>}

		\item  \xml{<Etats>/<Detail\_des\_etats>/<Etat>/<Aff\_infobd\_ope>}

		\item  \xml{<Etats>/<Detail\_des\_etats>/<Etat>/<Aff\_exo\_ope>}
      
		\end{itemize} 

\item  \xml{<Etats>/<Detail\_des\_etats>/<Etat>/<Class\_ope>} indique comment seront triées les opérations à l'intérieur de chaque niveau de regroupement~:

		\begin{itemize}

		\item 0~=~classement par date~;

		\item 1~=~classement par numéro d'opération~; 

		\item 2~=~classement par tiers~;

		\item 3~=~classement par catégorie~;

		\item 4~=~classement par imputation budgétaire~;

		\item 5~=~classement par notes~;

		\item 6~=~classement par mode de paiement~;

		\item 7~=~classement par numéro de chèque~;

		\item 8~=~classement par pièce comptable~;

		\item 9~=~classement par références bancaires~;

		\item 10~=~classement par numéro de rapprochement.
		
		\end{itemize} 

\item  \xml{<Etats>/<Detail\_des\_etats>/<Etat>/<Aff\_titres\_col>}~: 0~=~n'affiche pas les titres des colonnes, 1~=~les affiche~;

\item  \xml{<Etats>/<Detail\_des\_etats>/<Etat>/<Aff\_titres\_chgt>}~: 0~=~n'affiche pas les titres à chaque changement de section, 1~=~les affiche~;

\item  \xml{<Etats>/<Detail\_des\_etats>/<Etat>/<Pas\_detail\_ventil>}~: 0~=~affiche le détail des opérations ventilées, 1~=~ne l'affiche pas~;

\item  \xml{<Etats>/<Detail\_des\_etats>/<Etat>/<Sep\_rev\_dep>}~: 0~=~ne sépare pas les opérations de recette et de dépense, 1~=~les sépare~;

\item  \xml{<Etats>/<Detail\_des\_etats>/<Etat>/<Devise\_gen>}~: numéro de la devise générale de l'état~;

\item  \xml{<Etats>/<Detail\_des\_etats>/<Etat>/<Incl\_tiers>}~: 0~=~la liste des tiers de cet état n'apparaît pas dans la liste déroulante des tiers du formulaire de saisie des opérations, 1~=~elle apparaît~;

\item  \xml{<Etats>/<Detail\_des\_etats>/<Etat>/<Ope\_click>}~: 0~=~les opérations ne sont pas interactives, 1~=~elles le sont~;

\item  \xml{<Etats>/<Detail\_des\_etats>/<Etat>/<Exo\_date>}~: 0~=~utilise une plage de dates, 1~=~utilise les exercices~;

\item  \xml{<Etats>/<Detail\_des\_etats>/<Etat>/<Detail\_exo>}~: indique quel exercice est utilisé~:

	\begin{itemize}

	\item 0~=~tous les exercices~;

	\item 1~=~l'exercice courant~;

	\item 2~=~l'exercice précédent~;

	\item 3~=~exercices personnalisés.

	\end{itemize} 

\item  \xml{<Etats>/<Detail\_des\_etats>/<Etat>/<No\_exo>}~: si les exercices sont personnalisés, indique la liste des exercices utilisés dans l'état sous forme de suite de nombres séparés par des tirets~;

\item  \xml{<Etats>/<Detail\_des\_etats>/<Etat>/<Plage\_date>}~: indique quelle plage de dates est utilisée~:

	\begin{itemize}

	\item 0~=~toutes les dates~;

	\item 1~=~dates personnalisées~;

	\item 2~=~cumul à ce jour~;

	\item 3~=~mois en cours~;

	\item 4~=~année en cours~;

	\item 5~=~cumul mensuel~;

	\item 6~=~cumul annuel~;

	\item 7~=~mois précédent~;

	\item 8~=~année précedente ~;

	\item 9~=~30 derniers jours~;

	\item 10~=~3 derniers mois~;

	\item 11~=~6 derniers mois~;

	\item 12~=~12 derniers mois.

	\end{itemize} 

\item  \xml{<Etats>/<Detail\_des\_etats>/<Etat>/<Date\_debut>}~: date de début pour les dates personnalisées~;

\item  \xml{<Etats>/<Detail\_des\_etats>/<Etat>/<Date\_fin>}~: date de fin pour les dates personnalisées~;

\item  \xml{<Etats>/<Detail\_des\_etats>/<Etat>/<Utilise\_plages>}~: 0~=~ne sépare pas les résultats par période, 1~=~les sépare~;

\item  \xml{<Etats>/<Detail\_des\_etats>/<Etat>/<Sep\_plages>}~: 0~=~résultats séparés par semaines, 1~=~résultats séparés par mois, 2 =résultats séparés par année~;

\item  \xml{<Etats>/<Detail\_des\_etats>/<Etat>/<Deb\_sem\_plages>}~: jour où la semaine commence~: 0=lundi, etc.

\item  \xml{<Etats>/<Detail\_des\_etats>/<Etat>/<Detail\_comptes>}~: 0~=~ne sélectionne pas les opérations en fonction du compte, 1~=~sélectionne en fonction du compte~;

\item  \xml{<Etats>/<Detail\_des\_etats>/<Etat>/<No\_comptes>}~: liste des comptes utilisés séparés par des /~; 

\item  \xml{<Etats>/<Detail\_des\_etats>/<Etat>/<Grp\_ope\_compte>}~: 0~=~ne regroupe pas les opérations par comptes, 1~=~les regroupe~;

\item  \xml{<Etats>/<Detail\_des\_etats>/<Etat>/<Total\_compte>}~: 0~=~n'affiche pas de total lors d'un changement de compte, 1~=~affiche un total~;

\item  \xml{<Etats>/<Detail\_des\_etats>/<Etat>/<Aff\_nom\_compte>}~: 0~=~n'affiche pas le nom du compte, 1~=~l'affiche~;

\item  \xml{<Etats>/<Detail\_des\_etats>/<Etat>/<Type\_vir>}~: détermine si les opérations de virement sont sélectionnées pour l'état ou non~:

	\begin{itemize}

	\item 0~=~les virements ne sont pas inclus~;

	\item 1~=~seuls les virements vers les comptes d'actif ou de passif sont inclus~;

	\item 2~=~seuls les virements de ou vers les comptes ne figurant pas dans l'état sont inclus~;

	\item 3~=~les comptes dont les virements sont inclus sont sélectionnés manuellement.

	\end{itemize} 

\item  \xml{<Etats>/<Detail\_des\_etats>/<Etat>/<No\_comptes\_virements>}~: suite de nombres, séparés par des barres obliques, représentant les numéros des comptes dont les virements sont inclus dans l'état~;

\item  \xml{<Etats>/<Detail\_des\_etats>/<Etat>/<Exclure\_non\_vir>}~: 0~=~les opérations qui ne sont pas des virements ne sont pas exclues de l'état, 1~=~elles le sont~;

\item  \xml{<Etats>/<Detail\_des\_etats>/<Etat>/<Categ>} ~: 0~=~les opérations ne sont pas sélectionnées en fonction de la catégorie, 1~=~elles le sont~;

\item  \xml{<Etats>/<Detail\_des\_etats>/<Etat>/<Detail\_