%%%%%%%%%%%%%%%%%%%%%%%%%%%%%%%%%%%%%%%%%%%%%%%%%%%%%%%%%%%%%%%%%
% Contents : The financialyear chapter
% $Id : grisbi-manuel-financialyear.tex, v 0.6.0 2011/11/17 Jean-Luc Duflot
% some of its content was in tips chapter  : 
% $Id : grisbi-manuel-tips.tex, v 0.4 2002/10/27 Daniel Cartron
% $Id : grisbi-manuel-financialyear.tex, v 0.8.9 2012/04/27 Jean-Luc Duflot
% $Id : grisbi-manuel-financialyear.tex, v 1.0 2014/02/12 Jean-Luc Duflot
%%%%%%%%%%%%%%%%%%%%%%%%%%%%%%%%%%%%%%%%%%%%%%%%%%%%%%%%%%%%%%%%%

\chapter{Exercices\label{financialyear}}


Un exercice est une période d'une durée d'un an pour laquelle sont établies des prévisions financières ou sont dégagés des résultats financiers pour un agent économique (un individu, un ménage, une entreprise, un État, etc.).

Cette période ne correspond pas toujours à l'année civile. Le découpage du temps en exercices financiers implique, en matière de calcul des résultats, un certain nombre de régularisations avant de procéder à la clôture des comptes. Il arrive en effet que certains produits et charges enregistrés au cours de l'exercice concernent effectivement les exercices suivants et, réciproquement, que des produits et des charges relatifs à l'exercice en cours n'aient pas encore été enregistrés.

Vous pouvez utiliser les exercices si vous voulez distinguer vos opérations dans vos états, non par rapport à la date sous laquelle vous les enregistrez, mais par rapport à la période comptable à laquelle elles appartiennent.


\section{Exemples d'utilisation}


Prenons l'exemple d'un remboursement de Sécurité Sociale, qui intervient nécessairement avec un délai par rapport à la dépense. Si vous voulez faire un bilan objectif de vos dépenses de santé, il vous faudra le prendre en compte pendant la même année que la dépense, donc fausser la date de  remboursement, même s'il intervient l'année suivante. 

On peut aussi prendre l'exemple d'une facture téléphonique, souvent bimestrielle, couvrant la période du 11 décembre 2012 au 10 février 2013. Si vous enregistrez cette facture en 2013, la totalité de son montant sera affectée à cette année-là. Vous pouvez considérer que cela compense le déséquilibre induit par la facture de l'année précédente à la même époque, et que bon an mal an vos comptes sont à peu près bons.

La seule façon de procéder correctement est d'utiliser les exercices. La période courante d'un exercice est l'année civile, mais l'année scolaire peut être plus intéressante si vous êtes trésorier d'une association. Vous pouvez même faire des exercices mensuels, bien que comptablement ce soit une \emph{bizarrerie}.

Pour le remboursement de Sécurité Sociale, vous l'enregistrerez alors avec le même exercice que celui de la dépense, et votre état ou votre budget prendra en compte cette recette comme il le faut.

Pour la facture téléphonique, vous ventilerez celle-ci \emph{prorata temporis} (c'est-à-dire proportionnellement au temps). Pour simplifier, il est communément admis qu'une année se compose de 12 mois de 30 jours chacun. On a donc ici vingt jours sur l'exercice 2012, et quarante jours sur l'exercice 2013. Votre facture devra donc être ventilée en vingt soixantièmes sur l'exercice 2012, et quarante soixantièmes sur l'exercice 2013.


\section{Mise en place des exercices\label{financialyear-start}}


Avant de procéder à l'utilisation d'exercices, vous devez définir les aspects suivants :

\begin{itemize}
	\item la définition de l'exercice pour l'importation de fichiers de compte, dans le menu \menu{Édition - Préférences} : voir le paragraphe \vref{setup-general-import-financialyear}, \menu{Définition de l'exercice} ;	
	\item la détermination automatique de l'exercice dans le formulaire de saisie des opérations, dans le menu \menu{Édition - Préférences} : voir la section \vref{setup-form-behaviour}, \menu{Comportement} ;
	\item l'affichage du champ \menu{Exercice} dans le formulaire de saisie des opérations, dans le menu \menu{Édition - Préférences} : voir la section \vref{setup-form}, \menu{Formulaire des opérations} ;
	\item l'affichage du champ \menu{Exercice} dans la liste des opérations : voir le paragraphe \vref{transactions-list-fields}, \menu{Champs d'information et de saisie}.
\end{itemize}

% espace pour changement de thème
\vspacepdf{5mm}
Lors de la saisie d'une opération, vous pourrez alors, dans le champ \menu{Exercice} du formulaire de saisie, entrer l'exercice au clavier, ou le  choisir grâce à la liste déroulante commençant par \menu{Automatique}. Une fois l'opération validée, elle apparaîtra dans la liste des opérations avec son exercice, dans la cellule que vous avez choisie.

Vous pourrez aussi modifier l'ordre d'affichage des opérations par un \gls{tri} suivant les exercices ou les dates (voir le paragraphe \vref{transactions-list-sorts}, \menu{Tris}).


\section{Création, modification et suppression d'un exercice}


La création, la modification et la suppression d'un exercice se fait grâce au menu \menu{Édition - Préférences} (voir la section \vref{setup-resources-financialyear}, \menu{Exercices}).














