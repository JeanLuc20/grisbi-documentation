%%%%%%%%%%%%%%%%%%%%%%%%%%%%%%%%%%%%%%%%%%%%%%%%%%%%%%%%%%%%%%%%%
% Contents: The problems chapter
% $Id: grisbi-manuel-problems.tex, v 0.4 2002/10/27 Daniel Cartron
% $Id: grisbi-manuel-problems.tex, v 0.6.0 2011/11/17 Jean-Luc Duflot
% $Id: grisbi-manuel-problems.tex, v 0.6.0 2011/XX/XX Jean-Luc Duflot
%%%%%%%%%%%%%%%%%%%%%%%%%%%%%%%%%%%%%%%%%%%%%%%%%%%%%%%%%%%%%%%%%


\chapter{Problèmes, alertes et erreurs\label{problems}}

Un certain nombre de problèmes et erreurs prévisibles sont gérés directement par
Grisbi qui vous indique alors la nature du problème ou de l'erreur et vous
demande éventuellement d'intervenir. Certains messages sont aussi tout
simplement informatifs et ne seront pas traités ici.

\section{Problèmes connus\label{problems-known} }

\subsection{Permissions\label{problems-known-rights} }

Si vous êtes seul à utiliser votre fichier de comptes, l'idéal est que les permissions de votre fichier de compte soient à 600\footnote{ou encore -rw-~---~---},
c'est-à-dire que vous êtes le seul à pouvoir le lire et y écrire. Si vous êtes plusieurs à utiliser le fichier, l'idéal est alors de créer un groupe \file{grisbi}, de mettre les permissions du fichier de comptes à 660\footnote{ou encore -rw-~rw-~---}, et d'ajouter tous les utilisateurs autorisés dans ce groupe. Le fichier de comptes doit bien évidemment appartenir au groupe \file{grisbi}.
Grisbi vérifie à l'ouverture du fichier quelles sont les permissions et si elles ne sont pas égales à 600 il vous en avertit en affichant le message suivant: 

\begin{center}
\menu{Le fichier de comptes est lisible par tous}

\menu{(Ce message peut être désactivé dans le menu Configuration/Général.)} 
\end{center}

Cependant, dans le cadre d'une utilisation multi-utilisateurs, les permissions du
ficher doivent être à 660.

Si vous souhaitez avoir un fichier accessible par plusieurs utilisateurs, vous
aurez avantage à désactiver l'affichage de ce message avec le menu
\menu{Edition - Préférences - Affichage - /Messages et Alertes } (voir le paragraphe \vref{setup-display-messages}).

\subsection{Fichier de compte invalide}

Si Grisbi affiche le message suivant: 

\begin{center}
\menu{Fichier de compte invalide}
\end{center}

c'est que le fichier que vous essayez d'ouvrir n'a pas été créé par Grisbi.

\subsection{Version inconnue}

Si Grisbi affiche le message suivant: 

\begin{center}
\menu{Version inconnue}
\end{center}

c'est que le fichier que vous essayez d'ouvrir est bien un fichier Grisbi, mais d'une
version non compatible avec la version de Grisbi que vous utilisez. En général,
la version du fichier est supérieure à celle de Grisbi, sinon vous aurez plutôt 
le message ci-après.

\subsection{Fichier inférieur à la version 0.n.n}

Si la version de votre fichier de compte est trop ancienne, Grisbi affiche 
le message suivant: 

\begin{center}
\menu{Problème dans la récupération des opérations!}
\end{center}

En effet, Grisbi ne peut importer que les fichiers de la version précédente. Ce message
signifie que le fichier est trop vieux et qu'il vous faut passer par toutes les 
versions intermédiaires pour le transformer.

\subsection{Forcer l'enregistrement}

Le mode multi-utilisateurs de Grisbi permet uniquement une utilisation
successive et non simultanée du fichier de comptes. Pour indiquer que le fichier
est déjà ouvert par un utilisateur Grisbi positionne un drapeau à 1 dans
le fichier.

Si vous voulez ouvrir un fichier dans ce cas, Grisbi affiche le message suivant: 

\begin{center}
\menu{Attention le fichier semble déjà ouvert par un autre utilisateur ou n'a
pas été fermé correctement (plantage?).}

Vous ne pourrez enregistrer vos modifications qu'en activant l'option «Forcer
l'enregistrement» dans le menu Édition - Préférences - Généralités - Fichiers.
\end{center}

Vous pourrez ouvrir le fichier en consultation mais il est très fortement
déconseillé d'y faire quelque modification que ce soit.

Si vous avez activé l'option \menu{Forcer l'enregistrement}, à chaque fois que
vous ouvrirez le fichier Grisbi affichera le message suivant: 

\begin{center}
\menu{Attention le fichier semble déjà ouvert par un autre utilisateur ou n'a
pas été fermé correctement (plantage?).}

Cependant Grisbi va forcer l'enregistrement; cette option est déconseillée

sauf si vous êtes sûr que personne d'autre n'utilise le fichier pour le
moment.

La désactivation de cette option s'effectue dans le menu Édition - Préférences - Généralités - Fichiers.
\end{center}

Si l'option n'est pas activée, le menu \menu{Fichier/Enregistrer} ne sera pas
actif et vous ne pourrez pas enregistrer.

Il peut aussi se produire que Grisbi se plante et dans ce cas le drapeau n'a pas
pu se repositionner à 0. Dans ce cas, m\^eme si vous \^etes le seul utilisateur,
vous pouvez forcer l'enregistrement sans crainte. C'est la seule façon de
rétablir le drapeau dans un état correct. N'oubliez pas de revenir désactiver
l'option ensuite si vous n'utilisez pas Grisbi en mode multi-utilisateurs.

\section{Problèmes non connus}

Soit ces problèmes sont déjà connus et vous consulterez avec profit la F.A.Q.
sur le site de Grisbi, soit ils ne le sont pas encore et dans ce cas nous vous
remercions de nous transmettre un rapport de bug sur la \lang{liste de  
développement}\footnote{\urlListDevelEmail{}} ou mieux sur le \lang{gestionnaire de
bogues}\footnote{\urlBugTracker{}} du site de Grisbi.