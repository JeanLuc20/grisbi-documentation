%%%%%%%%%%%%%%%%%%%%%%%%%%%%%%%%%%%%%%%%%%%%%%%%%%%%%%%%%%%%%%%%%
% Contents: The FDL chapter
% $Id: grisbi-manuel-FDL.tex, v 0.4 2002/10/27 Daniel Cartron
%%%%%%%%%%%%%%%%%%%%%%%%%%%%%%%%%%%%%%%%%%%%%%%%%%%%%%%%%%%%%%%%%

\chapter{Licence de Documentation Libre GNU\label{gfdl}}

\section{Introduction}

Voici une adaptation non officielle de la Licence de Documentation
Libre du projet GNU. Elle n'a pas été publiée par la Free Software
Foundation et son contenu n'a aucune portée légale car seule la version
anglaise de ce document détaille le mode de distribution des logiciels
sous GNU FDL. Nous espérons cependant qu'elle permettra aux francophones
de mieux comprendre la FDL. 

\section{Licence de Documentation libre GNU Version 1.1, Mars 2000}

Copyright © Free Software Foundation, Inc.

59 Temple Place, Suite 330, Boston, MA 02111-1307 

États-Unis, 1989, 2000. 

La copie et la distribution de copies exactes de ce document
sont autorisées, mais aucune modification n'est permise. 

\section{Préambule} 

Le but de la présente Licence est de «libérer» un ouvrage,
un manuel, ou tout autre document écrit: assurer à chacun la liberté
véritable et complète de le copier et de le redistribuer, en le modifiant
ou non, commercialement ou non. De plus, la présente Licence garantit
à l'auteur et à l'éditeur un moyen d'être remerciés pour leur travail,
sans devoir assumer la responsabilité de modifications effectuées
par des tiers. 

La présente Licence est une variété de gauche d'auteur (\emph{copyleft}),
ce qui signifie que les travaux dérivés du document protégé doivent
être libres dans la même acception du mot « libre ». Elle complète
la Licence Publique Générale GNU ( GNU General Public License), qui
est une licence de gauche d'auteur conçue pour le logiciel libre. 

Nous avons conçu la présente licence dans le dessein de
l'utiliser pour les manuels de logiciels libres, car les logiciels
libres requièrent une documentation libre: un programme libre devrait
être accompagné de manuels offrant la même liberté que le programme
lui-même. Mais la présente Licence n'est pas limitée aux manuels de
logiciels; on peut l'utiliser pour tout travail textuel, indépendamment
du sujet, de son contenu, et de son mode de distribution (livre imprimé
ou autres). Nous recommandons la présente Licence principalement pour
les travaux à vocation d'instruction ou de référence. 


\section{Domaine d'application et définitions} 

La présente Licence s'applique à tout manuel ou travail
contenant une mention placée par le détenteur du copyright indiquant
que le document peut être distribué selon les termes de la présente
Licence. Le terme «Document», ci-dessous, se réfère à tout manuel
ou travail remplissant cette condition. Tout membre du public est
bénéficiaire de la licence et se trouve ici désigné par «vous». 

Une «Version Modifiée» du Document signifie: tout travail
contenant le Document, en intégralité ou en partie, aussi bien une
copie verbatim ou avec des modifications qu'une traduction dans une
autre langue. 

Une «Section Secondaire» est une annexe ou un avant-propos
du Document qui concerne exclusivement le rapport de l'éditeur ou
des auteurs du Document avec le sujet général du Document (ou des
domaines voisins) et ne contient rien qui puisse tomber directement
sous le coup du sujet général (par exemple, si le Document est en
quelque partie un manuel de mathématiques, une Section Secondaire
n'enseignera pas les mathématiques). Le rapport peut être une connexion
historique avec le sujet ou des domaines voisins, ou une précision
légale, commerciale, philosophique, éthique ou politique les concernant. 

Les «Sections Invariables» sont certaines Sections Secondaires
désignées par leurs titres comme Sections Invariables dans la mention
qui indique que le Document est couvert par la présente Licence. 

Les «Textes de Couverture» sont certains courts passages
du texte listés comme «Textes de Première de Couverture» ou «Textes
de Quatrième de Couverture» dans la mention qui indique que le Document
est couvert par la présente Licence. 

Une copie «Transparente» du Document signifie: une copie
lisible par une machine, réalisée dans un format dont les spécifications
sont disponibles au grand public, et dont le contenu peut être directement
visualisé et édité avec des éditeurs de texte génériques ou (pour
les images composées de pixels) avec des programmes de composition
d'images génériques ou (pour les figures techniques) un éditeur de
dessin vectoriel largement disponible, et qui soit approprié aux logiciels
qui mettent le texte en forme et le calibrent (formateurs de texte)
ou au transcodage automatique vers un assortiment de formats appropriés
aux formateurs de texte. Une copie réalisée dans un format de fichier
habituellement Transparent mais dont le balisage a été conçu pour
contrecarrer ou décourager des modifications ultérieures par le lecteur
n'est pas Transparente. Une copie qui n'est pas «Transparente» est
appelée «Opaque». 

Les formats appropriés aux copies Transparentes sont par
exemple l'ASCII brut sans balises, le format Texinfo, le format \LaTeX{},
SGML ou XML utilisant une DTD publiquement disponible, et l'HTML simple
et conforme à la norme, conçu en vue d'une modification manuelle.
Les formats Opaques incluent PostScript, PDF, les formats propriétaires
qui ne peuvent être lus et édités que par des traitements de texte
propriétaires, SGML et XML dont les DTD et/ou les outils de rendu
ne sont pas généralement disponibles, et l'HTML généré automatiquement
par certains traitements de texte à seule fin d'affichage. 

La «Page de Titre» désigne, pour un livre imprimé, la page
de titre proprement dite, plus les pages suivantes qui sont nécessaires
pour faire figurer, lisiblement, les éléments dont la présente Licence
requiert qu'ils apparaissent dans la Page de Titre. Pour les travaux
dont le format ne comporte pas de page de titre en tant que telle,
«Page de Titre» désigne le texte jouxtant l'apparition la plus marquante
du titre de ce travail, qui précède le début du corps du texte. 


\section{Copies VERBATIM} 

Vous pouvez copier et distribuer le Document sur tout support,
aussi bien commercialement que non, pour autant que la présente Licence,
les mentions de copyright, et les mentions de licence indiquant que
la présente Licence s'applique au Document soient reproduites sur
toutes les copies, et que vous n'ajoutiez aucune autre condition à
celles de la présente Licence. Vous ne pouvez pas user de moyens techniques
à des fins d'obstruction ou de contrôle de la lecture ou de la duplication
des copies que vous réalisez ou distribuez. Vous pouvez cependant
accepter des compensations en échange de la cession de copies. Si
vous distribuez un assez grand nombre de copies, vous devez aussi
suivre les conditions de la section Copies en quantité. 

Vous pouvez aussi prêter des copies, selon les mêmes conditions
que celles mentionnées ci-dessus, et vous pouvez exposer publiquement
des copies. 


\section{Copies en quantité} 

Si vous publiez des copies imprimées du Document à plus
de 100 exemplaires, et que la mention de la licence du Document exige
des Textes de Couverture, vous devez inclure les copies dans des couvertures
où figurent, clairement et lisiblement, tous ces Textes de Couverture:
les Textes de Première de Couverture sur la première de couverture,
et les Textes de Quatrième de Couverture sur la quatrième de couverture.
Les deux faces de la couverture doivent également clairement et lisiblement
vous identifier comme étant l'éditeur de ces copies. La première de
couverture doit présenter le titre complet, titre dont tous les mots
doivent être également mis en valeur et visibles. Vous pouvez ajouter
des éléments supplémentaires sur les couvertures. Toute copie avec
des changements limités aux couvertures, pour autant qu'ils préservent
le titre du Document et satisfont ces conditions, peut être considérée
comme une copie \emph{verbatim} à tous les autres égards. 

Si les textes destinés à l'une ou l'autre page de couverture
sont trop volumineux pour y figurer lisiblement, vous devez en mettre
les premiers (autant qu'il est raisonnablement possible) sur la couverture
proprement dite, et poursuivre sur les pages adjacentes. 

Si vous publiez ou distribuez des copies Opaques du Document
à plus de 100 exemplaires, vous devez soit inclure une copie Transparente
dans un format lisible par une machine, adapté au traitement automatisé,
en accompagnement de chaque copie Opaque, soit indiquer aux côtés
de ou dans chaque copie Opaque une adresse de réseau électronique
publiquement accessible, qui permette d'obtenir une copie Transparente
du Document, sans éléments ajoutés, à laquelle le grand public puisse
accéder pour téléchargement anonyme et sans frais en utilisant des
protocoles de réseau publics et standard. Si vous retenez la dernière
option, vous devez procéder prudemment et prendre les mesures nécessaires,
lorsque vous commencez la distribution de copies Opaques en quantité,
afin de vous assurer que cette copie Transparente demeurera accessible
au public pendant au moins une année après le moment de la distribution
(directement ou par l'intermédiaire de vos agents ou revendeurs) de
la dernière copie Opaque de cette édition. 

Il est souhaité, mais non exigé, que vous contactiez les
auteurs du Document bien avant la redistribution de tout grand nombre
de copies, afin de leur laisser la possibilité de vous fournir une
version mise à jour du Document. 


\section{Modifications} 

Vous pouvez copier et distribuer une Version Modifiée du
Document selon les conditions des sections Copies verbatim et Copies
en quantité qui précèdent, pourvu que vous diffusiez la Version Modifiée
sous couvert précisément de la présente Licence, avec la Version Modifiée
remplissant alors le rôle du Document, et ainsi autoriser la distribution
et la modification de la Version Modifiée à quiconque en possède une
copie. En complément, vous devez accomplir ce qui suit sur la Version
Modifiée: 

A. Utilisez dans la Page de Titre (et sur les couvertures,
le cas échéant) un titre distinct de celui du Document et de ceux
des précédentes versions (qui doivent, s'il en existe, être citées
dans la section «Historique» du Document). Vous pouvez utiliser le
même titre qu'une version précédant la vôtre si l'éditeur original
vous en donne la permission. 

B. Indiquez sur la Page de Titre, comme auteurs, une ou
plusieurs personnes ou entités responsables de l'écriture des modifications
de la Version Modifiée, ainsi qu'au moins cinq des principaux auteurs
du Document (ou tous les auteurs principaux, s'ils sont moins de cinq). 

C. Apposez sur la Page de Titre de nom de l'éditeur de la
Version Modifiée, en tant qu'éditeur. 

D. Préservez toutes les mentions de copyright du Document. 

E. Ajoutez une mention de copyright appropriée à vos modifications,
aux côtés des autres mentions de copyright. 

F. Incluez, immédiatement après les mentions de copyright,
une mention de licence qui accorde la permission publique d'utiliser
la Version Modifiée selon les termes de la présente Licence, sous
la forme présentée dans la section Addendum ci-dessous. 

G. Préservez dans cette mention de licence les listes complètes
des Sections Invariables et des Textes de Couverture exigés, données
dans la mention de licence du Document. 

H. Incluez une copie non altérée de la présente Licence. 

I. Préservez la section intitulée «Historique», et son titre,
et ajoutez-y un article indiquant au moins le titre, l'année, les
nouveaux auteurs, et l'éditeur de la Version Modifiée telle qu'elle
apparaît sur la Page de Titre. Si le Document ne contient pas de section
intitulée «Historique», créez-en une et indiquez-y le titre, l'année,
les auteurs et l'éditeur du Document tels qu'indiqués sur la Page
de Titre, puis ajoutez un article décrivant la Version Modifiée, comme
exposé dans la phrase précédente. 

J. Préservez, le cas échéant, l'adresse de réseau électronique
donnée dans le Document pour accéder publiquement à une copie Transparente
du Document, et préservez de même les adresses de réseau électronique
données dans le Document pour les versions précédentes, sur lequelles
le Document se fonde. Cela peut être placé dans la section « Historique ».
Vous pouvez omettre l'adresse de réseau électronique pour un travail
qui a été publié au moins quatre ans avant le Document lui-même, ou
si l'éditeur original de la version à laquelle il se réfère en donne
l'autorisation. 

K. Dans toute section intitulée «Remerciements» ou «Dédicaces»,
préservez le titre de section et préservez dans cette section le ton
et la substance de chacun des remerciements et/ou dédicaces donnés
par les contributeurs. 

L. Préservez toutes les Sections Invariables du Document,
non altérées dans leurs textes et dans leurs titres. Les numéros de
sections ou leurs équivalents ne sont pas considérés comme faisant
partie des titres de sections. 

M. Supprimez toute section intitulée «Approbations». Une
telle section ne doit pas être incluse dans la Version Modifiée. 

N. Ne changez pas le titre d'une section existante en «Approbations»
ou en un titre qui entre en conflit avec celui d'une Section Invariable
quelconque. 

Si la Version Modifiée inclut de nouvelles sections d'avant-propos
ou des annexes qui remplissent les conditions imposées aux Sections
Secondaires et ne contiennent aucun élément tiré du Document, vous
pouvez, à votre convenance, désigner tout au partie de ces sections
comme «Invariables». Pour ce faire, ajoutez leurs titres à la liste
des Sections Invariables dans la mention de licence de la Version
Modifiée. Ces titres doivent être distincts de tout autre titre de
section. 

Vous pouvez ajouter une section intitulée «Approbations»,
pourvu qu'elle ne contienne rien d'autre que l'approbation de votre
Version Modifiée par diverses parties -- par exemple, indication d'une
revue par les pairs ou bien que le texte a été approuvé par une organisation
en tant que définition de référence d'un standard. 

Vous pouvez ajouter un passage de cinq mots ou moins en
tant que Texte de la Première de Couverture, et un passage de 25 mots
ou moins en tant que Texte de Quatrième de Couverture, à la fin de
la liste des Textes de Couverture de la Version Modifiée. Toute entité
peut ajouter (ou réaliser, à travers des arrangements) au plus un
passage en tant que Texte de la Première de Couverture et au plus
un passage en tant que Texte de la Quatrième de Couverture. Si le
Document inclut déjà un texte de Couverture pour la même couverture,
précédemment ajouté par vous ou, selon arrangement, réalisé par l'entité
pour le compte de laquelle vous agissez, vous ne pouvez en ajouter
un autre; mais vous pouvez remplacer l'ancien, avec la permission
explicite de l'éditeur qui l'a précédemment ajouté. 

Le ou les auteur(s) et le ou les éditeur(s) du Document
ne confèrent pas par la présente Licence le droit d'utiliser leur
nom à des fins publicitaires ou pour certifier ou suggérer l'approbation
de n'importe quelle Version Modifiée. 


\section{Mélange de documents} 

Vous pouvez mêler le Document à d'autres documents publiés
sous la présente Licence, selon les termes définis dans la section
Modifications ci-dessus, traitant des versions modifiées, pour autant
que vous incluiez dans ce travail toutes les Sections Invariables
de tous les documents originaux, non modifiées, et en les indiquant
toutes comme Sections Invariables de ce travail dans sa mention de
licence. 

Le travail issu du mélange peut ne contenir qu'une copie
de cette Licence, et de multiples Sections Invariables identiques
peuvent n'être présentes qu'en un exemplaire qui les représentera
toutes. S'il existe plusieurs Sections Invariables portant le même
nom mais des contenus différents, faites en sorte que le titre de
chacune de ces sections soit unique, en indiquant à la fin de chacune
d'entre elles, entre parenthèses, le nom de l'auteur original ou de
l'éditeur de cette section s'il est connu, ou un numéro unique dans
les collisions restantes. Pratiquez les mêmes ajustements pour les
titres de sections, dans la liste des Sections Invariables de la mention
de licence de ce travail mélangé. 

Dans le mélange, vous devez regrouper toutes les sections
intitulées «Historique» dans les divers documents originaux, afin
de constituer une unique section intitulée «Historique»; combinez
de même toutes les sections intitulée «Remerciements», et toutes les
sections intitulées «Dédicaces». Vous devez supprimer toutes les sections
intitulées «Approbations». 


\section{Recueil de documents} 

Vous pouvez réaliser un recueil regroupant le Document et
d'autres documents publiés sous la présente Licence, et remplacer
les diverses copies de la présente Licence figurant dans les différents
documents par une copie unique incluse dans le recueil, pour autant
que vous suiviez les règles de la présente Licence relatives à la
copie \emph{verbatim} pour chacun de ces documents, dans tous les
autres aspects. 

Vous pouvez n'extraire qu'un seul document d'un tel recueil,
et le distribuer individuellement sous la présente Licence, pour autant
que vous insériez une copie de la présente Licence dans le document
extrait, et que vous suiviez la présente Licence dans tous ses autres
aspects concernant la reproduction verbatim de ce document. 


\section{Agrégation avec des travaux indépendants} 

Une compilation du Document ou de ses dérivés avec d'autres
documents ou travaux séparés et indépendants, ou bien sur une unité
de stockage ou un support de distribution, ne compte pas comme une
Version Modifiée de ce Document, pour autant qu'aucun copyright de
compilation ne soit revendiqué pour la compilation. Une telle compilation
est appelée une «agrégation», et la présente Licence ne s'applique
pas aux autres travaux contenus et ainsi compilés avec le Document,
sous prétexte du fait qu'ils sont ainsi compilés, s'ils ne sont pas
eux-mêmes des travaux dérivés du Document. 

Si les exigences de la section Copies en quantité en matière
de Textes de Couverture s'appliquent aux copies du Document, et si
le Document représente moins du quart de la totalité de l'agrégat,
alors les Textes de Couverture du Document peuvent n'être placés que
sur les couvertures qui entourent le document, au sein de l'agrégation.
Dans le cas contraire, ils doivent apparaître sur les couvertures
entourant tout l'agrégat. 

La traduction est considérée comme un type de modification,
de sorte que vous devez distribuer les traductions de ce Document
selon les termes de la section Modifications. La substitution des
Sections Invariables par des traductions requiert une autorisation
spéciale de la part des détenteurs du copyright, mais vous pouvez
ajouter des traductions de tout ou partie des Sections Invariables
en sus des versions originales de ces Sections Invariables. Vous pouvez
inclure une traduction de la présente Licence pourvu que que vous
incluiez la version originale, en anglais, de la présente Licence.
En cas de désaccord entre la traduction et la version originale, en
anglais, de la présente Licence, la version originale prévaudra. 


\section{Révocation} 

Vous ne pouvez copier, modifier, sous-licencier ou distribuer
le Document autrement que selon les conditions expressément prévues
par la présente Licence. Toute tentative de copier, modifier, sous-licencier
ou distribuer autrement le Document est nulle et non avenue, et supprimera
automatiquement vos droits relatifs à la présente Licence. De même,
les parties qui auront reçu de votre part des copies ou des droits
sous couvert de la présente Licence ne verront pas leurs licences
révoquées tant que ces parties demeureront en pleine conformité avec
la présente Licence. 


\section{Révisions futures de la présente licence} 

La \emph{Free Software Foundation} («fondation du logiciel
libre») peut publier de nouvelles versions révisées de la présente
\emph{GNU Free Documentation License} de temps à autre. Ces nouvelles
versions seront similaires, dans l'esprit, à la présente version,
mais peuvent différer dans le détail pour prendre en compte de nouveaux
problèmes ou de nouvelles inquiétudes. Consultez http://www.gnu.org/copyleft/. 

Chaque version de la Licence est publiée avec un numéro
de version distinctif. Si le Document précise qu'une version particulière
de la présente Licence, «ou toute version postérieure» s'applique,
vous avez la possibilité de suivre les termes et les conditions aussi
bien de la version spécifiée que de toute version publiée ultérieurement
(pas en tant que brouillon) par la \emph{Free Software Foundation}.
Si le Document ne spécifie pas un numéro de version de la présente
Licence, vous pouvez choisir d'y appliquer toute version publiée (pas
en tant que brouillon) par la \emph{Free Software Foundation}. 


\section{ADDENDUM: Comment utiliser la présente licence dans vos documents} 

Pour utiliser la présente Licence dans un document que vous
avez rédigé, insérez une copie de la présente Licence dans le document
et placez le copyright et les mentions de licence suivants juste après
la page de titre: 

\begin{verbatim}

{Copyright (c) ANNÉE VOTRE NOM. }

{Permission est accordée de copier, distribuer et/ou 
modifier ce document selon les termes de la Licence 
de Documentation Libre GNU (GNU Free Documentation 
License), version 1.1 ou toute version ultérieure 
publiée par la Free Software Foundation; avec les 
Sections Invariables qui sont LISTE DES TITRES; 
avec les Textes de Première de Couverture qui sont 
LISTE, et avec les Textes de Quatrième de Couverture 
qui sont LISTE. Une copie de la présente Licence est
incluse dans la section intitulée « Licence de 
Documentation Libre GNU ». }

\end{verbatim}

Si vous n'avez pas de Sections Invariables, écrivez, 
«sans Sections Invariables» au lieu d'en indiquer la 
liste. Si vous n'avez pas de Textes de Première de 
Couverture, écrivez «sans Texte de Première de 
Couverture» au lieu de «les Textes de Quatrième de 
Couverture qui sont LISTE»; et de la même manière 
pour les Textes de Quatrième de Couverture.

Si votre document contient des exemples non triviaux
 de code de programmation, nous recommandons de diffuser 
 ces exemples en parallèle sous la licence libre de 
 votre choix, comme la Licence Publique Générale GNU 
 ( GNU General Public License), afin de permettre
leur utilisation dans des logiciels libres. 

