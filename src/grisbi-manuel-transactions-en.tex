%%%%%%%%%%%%%%%%%%%%%%%%%%%%%%%%%%%%%%%%%%%%%%%%%%%%%%%%%%%%%%%%%
% Contents : The transactions chapter
% $Id : grisbi-manuel-transactions.tex,v 0.4 2002/10/27 Daniel Cartron
% $Id : grisbi-manuel-transactions.tex, v 0.5.0 2004/06/01 Loic Breilloux
% some of its content was in tips chapter  : 
% $Id : grisbi-manuel-tips.tex, v 0.4 2002/10/27 Daniel Cartron
% $Id : grisbi-manuel-transactions.tex, v 0.6.0 2011/11/17 Jean-Luc Duflot
% some of its content was in tips chapter :
% $Id : grisbi-manuel-tips.tex, v 0.5.0 2004/06/01 Loic Breilloux
% $Id : grisbi-manuel-transactions.tex, v 0.8.9 2012/04/27 Jean-Luc Duflot
% $Id : grisbi-manuel-transactions.tex, v 1.0 2014/02/12 Jean-Luc Duflot
% some of its content was in accounts chapter :
% $Id : grisbi-manuel-accounts.tex, v 0.8.9 2012/04/27 Jean-Luc Duflot
%%%%%%%%%%%%%%%%%%%%%%%%%%%%%%%%%%%%%%%%%%%%%%%%%%%%%%%%%%%%%%%%%

\chapter{Opérations d'un compte\label{transactions}}

Note: This chapter still awaits translation...


\section{Barre d'outils\label{transactions-functions}}


\section{Liste des opérations\label{transactions-list}}


\subsection{Description\label{transactions-list-description}}



\subsection{Champs d'information et de saisie\label{transactions-list-fields}}


\subsection{Gestion des champs d'information\label{transactions-list-fields-manage}}

\subsubsection{Ajout d'un champ d'information\label{transactions-list-fields-add}}


\subsubsection{Modification d'un champ d'information\label{transactions-list-fields-modify}}


\subsubsection{Suppression d'un champ d'information\label{transactions-list-fields-remove}}


\subsubsection{Déplacement d'un champ d'information\label{transactions-list-fields-move}}


\subsection{Tris\label{transactions-list-sorts}}


\section{Formulaire de saisie\label{transactions-form}}


\section{Solde pointé\label{transactions-balance}}


\section{Sélection d'une opération \label{transactions-selection}}


\section{Saisie d'une nouvelle opération\label{transactions-new}}


\subsection{Commandes principales au clavier\label{transactions-new-keyboard}}


\subsection{Saisie de date au clavier\label{transactions-new-dates}}


\subsection{Saisie de date au calendrier\label{transactions-new-calendar}}


\subsection{Listes déroulantes\label{transactions-new-lists}}


\subsection{Saisie de formules\label{transactions-new-numbers}}

\subsection{Remise de chèques ou d'espèces\label{transactions-new-cheque}}


\subsection{Virements\label{transactions-new-transfer}}


\subsubsection{Virements externes\label{transactions-new-transfer-external}}

\subsubsection{Virements internes\label{transactions-new-transfer-internal}}

\subsubsection{Contre-opération d'un virement\label{transactions-new-transfer-associatedOperation}}

\subsection{Avances : consentement et réception\label{transactions-new-advance}}

\subsection{Rappel automatique d'une opération\label{transactions-new-recall}}


\section{Saisie d'une opération en devises\label{transactions-currencies}}


\section{Saisie d'une opération avec tiers virtuel\label{transactions-virtualThird}}


\section{Ventilation d'une opération\label{transactions-breakdown}}

\subsection{Saisie de l'opération ventilée\label{transactions-breakdown-master}}


\subsection{Saisie des sous-opérations ventilées\label{transactions-breakdown-slave}}


\subsection{Validation de l'opération ventilée\label{transactions-breakdown-validation}}


\subsection{Rappel automatique d'une opération ventilée\label{transactions-breakdown-recall}}


\section{Modification d'une opération\label{transactions-modify}}


\section{Suppression d'une opération\label{transactions-delete}}


\section{Opération sélectionnée comme modèle\label{transactions-model}}


\section{Clonage d'une opération\label{transactions-duplicate}}


\section{Conversion d'une opération en opération planifiée\label{transactions-schedule}}


\section{Déplacement d'une opération vers un autre compte\label{transactions-move}}


\section{Nouveau tiers, catégorie ou imputation budgétaire\label{transactions-fillcombo}}


\section{Impression de la liste des opérations\label{transactions-print}}


%XXXXXXXXXXXXXXXXXXXXXXXXXXXXXXXXXXXXXXXXXXXXXXXXXXXXXXXXXXXXXXXXXXXXXXXXXXXXXXXXXX
%
% [benj] supprimé car franchement, ça n'a rien à faire ici.
%
% PLEASE NOTE !  DO NOT TRANSLATE THE FOLLOWING SECTION
%\section{Quelques mots sur un anglicisme trop répandu}
%
%On voit régulièrement dans les documentations, livres et autres articles,
%publiés sur l'Internet et ailleurs, le mot \emph{complétion}. Ceci est un
%horrible anglicisme. 
%
%Que l'on assimile un mot anglais lorsque le mot français équivalent n'existe pas
%peut se comprendre, encore que le néologisme intelligent me semble préférable.
%Il ne s'agit bien sûr pas de tomber dans le ridicule de nos académiciens qui
%traduisent mail par mèl et CD-Rom par cédérom. Passons sur leur manque
%d'imagination alors qu'ils ont su par autrefois créer des mots comme disquette,
%informatique et ordinateur, qui n'existent nulle part ailleurs !
%
%Dans le cas précis du mot \emph{complétion} un traducteur du \emph{Guide de
%l'administration réseau sous linux,} Editions O'Reilly, avait cru bien faire en
%essayant de créer le mot \emph{complétement} (avec un accent aigu) qui voulait
%signifier \emph{action de compléter}. L'initiative était louable mais ô combien
%risquée \ldots En effet lorsque l'on s'essaye à ce genre d'exercice de style, il est
%prudent de maîtriser tous les rouages de la langue. Et en l'occurrence il existe
%une règle dite la \emph{règle de l'ernance} (page 56 du \emph{Bescherelle
%Orthographe}) \footnote{Si quelqu'un possède ce livre je serais très heureux
%qu'il me fasse parvenir le texte exact de cette règle. Je ne l'ai 
%malheureusement plus\ldots} qui explique parfaitement le mécanisme de
%changement d'accent entre un verbe et le substantif dérivé
%\footm{André Cianfarini, qui se définit lui-même comme \emph{un développeur autodidacte qui n'a jamais fini sa thèse de grammaire pour cause de relations passionnelles avec l'informatique} me communique les précisions suivantes : 
%\begin{quote}
%C'est un processus phonologique qui est à l'origine de ces
%transformations ; pour simplifier, on pourrait le résumer ainsi :
%un "e" fermé (l'avant-dernier "e" de "préférer" par exemple) devient un
%"e" ouvert (l'avant-dernier "e" de "je préfère" par exemple) lorsque la
%syllabe qui le suit est atone.
%\end{quote} 
%}. Si ce malheureux
%traducteur l'avait connue il n'aurait pas commis cet affreux barbarisme\ldots
%
%Et si avant toute chose ce traducteur avait eu l'idée de vérifier dans son
%dictionnaire que le mot n'existait pas déjà\ldots il aurait trouvé :
%
%\begin{description}
%
%\item [complètement]n.m. Action de compléter : \emph{le complètement
%d'un livre}. 
%
%\end{description}
%
%Et j'ai bien pris soin de chercher ce mot dans le plus vieil exemplaire du
%\emph{Larousse universel} en ma possession, soit l'édition de 1949 !
%
%Donc adoptons sans aucun remord le mot complètement, et, puisque cette action
%est automatique, allons sans crainte jusqu'à écrire auto-complètement, ce qui
%est parfaitement compréhensible, intuitif et 100\% français !
%
%Cocorico !
%
%Et puissent ces quelques lignes tordre définitivement le cou au mot complétion,
%cette \emph{affreuseté non-permissable} ! ;-)
%
%from now you can translate again
